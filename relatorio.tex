\documentclass[12pt,a4paper]{article}
\usepackage[utf8]{inputenc}
\usepackage[portuguese]{babel}
\usepackage[T1]{fontenc}
\usepackage{amsmath}
\usepackage{amsfonts}
\usepackage{amssymb}
\usepackage[left=3.0cm,right=3.0cm,top=3.5cm,bottom=2.5cm]{geometry}
\usepackage{makeidx}
\usepackage{graphicx}
\usepackage{fancyvrb}
\usepackage[dvipsnames]{xcolor}
\usepackage{xcolor}

\usepackage[paperwidth=841pt,paperheight=595pt,top=28pt,right=28pt,bottom=28pt,left=28pt, includefoot, includehead]{geometry}
\usepackage{listings}
\usepackage{textcomp}
\usepackage{color}

\definecolor{codegreen}{rgb}{0,0.6,0}
\definecolor{codegray}{rgb}{0.5,0.5,0.5}
\definecolor{codepurple}{HTML}{C42043}
\definecolor{backcolour}{HTML}{F2F2F2}
\definecolor{bookColor}{cmyk}{0,0,0,0.90}  
\color{bookColor}

\lstset{
breaklines=true,
basicstyle=\small\ttfamily,
columns=flexible,
}





\usepackage{mathrsfs}

\usepackage{fvextra} % loads also fancyvrb
\usepackage{xpatch}

\DeclareMathVersion{ttmath}
\DeclareSymbolFont{latinletters}{OT1}{\ttdefault}{m}{n}
%\SetSymbolFont{latinletters}{ttmath}{OT1}{\ttdefault}{m}{n}
\SetSymbolFont{letters}{ttmath}{OML}{ccm}{m}{it}
\SetSymbolFont{symbols}{ttmath}{OMS}{ccsy}{m}{n}
\SetSymbolFont{largesymbols}{ttmath}{OMX}{ccex}{m}{n}

\newcommand{\changeletters}{%
  \count255=`A
  \advance\count255 -1
  \loop\ifnum\count255<`Z
    \advance\count255 1
    \mathcode\count255=\numexpr\number\symlatinletters*256+\count255\relax
  \repeat
  \count255=`a
  \advance\count255 -1
  \loop\ifnum\count255<`z
    \advance\count255 1
    \mathcode\count255=\numexpr\number\symlatinletters*256+\count255\relax
  \repeat
  \count255=`0
  \advance\count255 -1
  \loop\ifnum\count255<`9
    \advance\count255 1
    \mathcode\count255=\numexpr\number\symlatinletters*256+\count255\relax
  \repeat
}

\xapptocmd{\ttfamily}{\mathversion{ttmath}\changeletters}{}{}


\usepackage{placeins}

\setcounter{tocdepth}{4}
\setcounter{secnumdepth}{4}

\usepackage{sectsty}
\allsectionsfont{\normalfont\scshape}


\input{structure.tex}


\begin{document}

% Aumenta o espaçamento entre as palavras
\spaceskip=1.5\fontdimen2\font plus 1.5\fontdimen3\font
minus 1.5\fontdimen4\font

\begin{titlepage}

\begin{center}

\textbf{UNIVERSIDADE FEDERAL DE SÃO CARLOS\\CAMPUS SOROCABA\\\vspace{3cm} BACHARELADO EM CIÊNCIA DA COMPUTAÇÃO\\\vspace{3cm}SISTEMAS DE BANCO DE DADOS\\
Prof. Sahudy Montenegro González\\\vspace{3cm}
PROJETO INTEGRADO\\Fase Intermediária\\\vspace{0.5cm}
BASE DE DADOS BRASILEIRA\\TEMA 4 - Cadastro Brasileiro de Escolas \\\vspace{4.0cm}
Bianca Gomes Rodrigues - 743512\\
Pietro Zuntini Bonfim - 743588\\
\\\vspace{3.5cm}
Sorocaba-SP\\21 de Abril de 2019}

\end{center}

\end{titlepage}

% INDICE

\pagebreak
\renewcommand*\contentsname{Índice}
\tableofcontents
\pagebreak

\section{Descrição do Mini-Mundo}

O objetivo deste projeto é criar uma aplicação que permita o armazenamento dos dados das escolas brasileiras do Brasil. O projeto, integrado com as disciplinas de \texttt{Desenvolvimento para Web} e \texttt{Sistemas de Bancos de Dados}, permitirá o gerenciamento e a visualização das escolas brasileiras.

\begin{info}[Sobre os dados:]
É importante ressaltar que os dados escolhidos para o cadastramento das escola foram baseado nos microdados fornecidos pelo INEP.
\end{info}


\section{Esquema do Banco de Dados}

Nesta seção será apresentado o diagrama que contém as tabelas e atributos do banco de dados, além do significado de cada um dos atributos. Todas as informações encontram-se a seguir na figura 1.

\begin{figure}[h]
    \centering
    \includegraphics[scale=0.35]{diagrama_escolas_brasileiras.png}
    \caption{Diagrama do BD}
    \label{fig:diagrama}
\end{figure}

\pagebreak


\begin{flushleft}
    \textbf{Informações Básicas sobre a Escola}
\end{flushleft}

\begin{itemize}

    \item \texttt{co\_escola}: Código da Escola

    \item \texttt{nome\_escola}: Nome da Escola

    \item \texttt{situacao\_funcionamento}: Situação de Funcionamento da Escola (Em Atividade, Paralisada ou Extinta)

    \item \texttt{inicio\_ano\_letivo}: Data de Início do Ano Letivo

    \item \texttt{termino\_ano\_letivo}: Data de Término do Ano Letivo
    
    \item \texttt{dependencia\_adm}: Tipo de Dependência Administrativa da Escola (Federal, Estadual, Municipal ou Privada)

    \item \texttt{regulamentada}: Se a Escola é regulamentada ou não

    \item \texttt{qtd\_salas\_existentes}: Número de salas existentes na escola

    \item \texttt{qtd\_salas\_utilizadas}: Número de salas sendo efetivamente utilizadas na escola

    \item \texttt{qtd\_funcionarios}: Número de funcionários da escola

\end{itemize}

\begin{flushleft}
    \textbf{Informações de Localização Escola}
\end{flushleft}

\begin{itemize}

    \item \texttt{co\_distrito}: Código Completo do Distrito da Escola

    \item \texttt{localizacao}: Área da Localização da Escola (Urbana ou Rural)

\end{itemize}


    
\begin{flushleft}
    \textbf{Informações Adicionais sobre a Escola}
\end{flushleft}

\begin{itemize}

    \item \texttt{agua\_filtrada}: Se a Escola possui água filtrada ou não

    \item \texttt{esgoto}: Se a Escola possui sistema de esgoto ou não

    \item \texttt{coleta\_de\_lixo}: Se a Escola possui sistema de coleta de lixo ou não

    \item \texttt{reciclagem}: Se a Escola possui sistema de reciclagem de lixo ou não

\end{itemize}

\begin{flushleft}
    \textbf{Informações de Dependências da Escola}
\end{flushleft}

\begin{itemize}

    \item \texttt{sala\_diretoria}: Se a Escola possui uma sala de diretoria ou não

    \item \texttt{sala\_professor}: Se a Escola possui salas de professor ou não

    \item \texttt{laboratorio\_informatica}: Se a Escola possui um laboratório de informática ou não

    \item \texttt{laboratorio\_ciencias}: Se a Escola possui um laboratório de ciências ou não

    \item \texttt{quadra\_esportes}: Se a Escola possui quadra de esportes ou não

    \item \texttt{cozinha}: Se a Escola possui cozinha ou não

    \item \texttt{biblioteca}: Se a Escola possui biblioteca ou não

    \item \texttt{sala\_leitura}: Se a Escola possui sala de leitura ou não

    \item \texttt{parque\_infantil}: Se a Escola possui parque infantil ou não

    \item \texttt{bercario}: Se a Escola possui berçário ou não

    \item \texttt{acessibilidade\_deficiencia}: Se a Escola possui dependências e vias adequadas à alunos com deficiência ou mobilidade reduzida ou não
    
    \item \texttt{secretaria}: Se a Escola possui secretaria ou não

    \item \texttt{refeitorio}: Se a Escola possui refeitório ou não

    \item \texttt{alimentacao}: Se a Escola oferece alimentação ou não

    \item \texttt{auditorio}: Se a Escola possui auditório ou não

    \item \texttt{alojamento\_alunos}: Se a Escola possui alojamento para alunos ou não

    \item \texttt{alojamento\_professores}: Se a Escola possui alojamento para professores ou não

    \item \texttt{area\_verde}: Se a Escola possui uma área verde ou não

    \item \texttt{internet}: Se a Escola possui acesso à internet ou não

\end{itemize}

\begin{flushleft}
    \textbf{Informações de Oferta de Matrícula}
\end{flushleft}

\begin{itemize}

    \item \texttt{creche}: Se a Escola oferece creche ou não

    \item \texttt{pre\_escola}: Se a Escola oferece pré-escola ou não

    \item \texttt{ens\_fundamental\_anos\_iniciais}: Se a Escola oferece Ensino Fundamental do 1º ao 5º ano ou não

    \item \texttt{ens\_fundamental\_anos\_finais}: Se a Escola oferece Ensino Fundamental do 5º ao 9º ano ou não

    \item \texttt{ens\_medio\_normal}: Se a Escola oferece Ensino Médio do 1º ao 3º ano ou não

    \item \texttt{ens\_medio\_integrado}: Se a Escola oferece Ensino Médio integrado com Curso Técnico ou não

\end{itemize}

\section{Especificação de Consultas}

Nesta seção serão especificadas as consultas que farão parte do projeto. É importante ressaltar que são \textbf{duas} consultas, com atributos relativos (expressões regulares) e absolutos (expressões exatas). As consultas definidas pelo grupo serão apresentadas a seguir.

\begin{enumerate}
    \item 
        Buscar todas as escolas disponíveis em um determinado \texttt{Estado} (UF) e \texttt{Munícipio} (do Estado previamente selecionado), especificando um trecho do \texttt{nome da Escola} (dentre as previamente selecionadas no Munícipio). Além disso, o resultado obtido será ranqueado pelo \texttt{número de funcionários} da Escola.
        
        \textbf{Campos de busca}: \texttt{UF} e \texttt{Município} (absolutos), e \texttt{Nome da Escola} (relativo). 
        
        \textbf{Campos de visualização do resultado}: Inicialmente \texttt{código da escola}, \texttt{nome}, \texttt{situação de funcionamento}, \texttt{dependência administrativa} e  \texttt{ofertas de matricula}. Posteriormente, será possível a visualização dos demais campos da escola.
    
        \textbf{Operadores das condições}: \texttt{nome} da escola (\texttt{ILIKE}) e os demais (\texttt{=})
        
        \textbf{SQL}:
        \begin{Verbatim}[commandchars=\\\{\}]
SELECT  e.co\_escola, e.nome\_escola, e.situacao\_funcionamento,
            e.dependencia\_adm, e.bercario, e.creche, e.pre\_escola,
            e.ens\_fundamental\_anos\_iniciais, e.ens\_fundamental\_anos\_finais,
            e.ens\_medio\_normal, e.ens\_medio\_integrado
FROM escola e
JOIN distrito d on e.co\_distrito = d.co\_distrito
JOIN municipio m on d.co\_municipio = m.co\_municipio
JOIN microrregiao mi on m.co\_microrregiao = mi.co\_microrregiao
JOIN mesorregiao me on mi.co\_mesorregiao = me.co\_mesorregiao
JOIN uf u on me.co\_uf = u.co\_uf
WHERE u.co\_uf = \textcolor{MidnightBlue}{<codigo_uf>}
AND m.co\_municipio = \textcolor{MidnightBlue}{<codigo_municipio>}
AND e.nome\_escola ILIKE '\%\textcolor{MidnightBlue}{<nome_escola>}\%'
ORDER BY e.qtd\_funcionarios;
        \end{Verbatim}
    
    \vspace{0.5cm}
    \item 
         Buscar o número de escolas com cada tipo de dependência administrativa: \texttt{Federal}, \texttt{Estadual}, \texttt{Municipal} e \texttt{Privada} em uma determinada \texttt{Região}. 
    
        \textbf{Campos de busca}: \texttt{Código da Região}.
        
        \textbf{Campos de visualização do resultado}: Quantidade de \texttt{Escolas} na \texttt{Região}, quantidade de \texttt{Escolas Federais}, quantidade de \texttt{Escolas Estaduais}, quantidade de \texttt{Escolas Municipais} e quantidade de \texttt{Escolas Privadas}.
        
        \textbf{Operadores das condições}: \texttt{Código da Região} (\texttt{=}).
        
        \textbf{SQL}:
        \begin{Verbatim}[commandchars=\\\{\}]
SELECT count(e.co_escola) as qtd_escolas, (
        SELECT count(e2.co_escola)
        FROM escola e2
        WHERE e2.dependencia_adm = 'Federal'
    ) as qtd_federal, (
        SELECT count(e3.co_escola)
        FROM escola e3
        WHERE e3.dependencia_adm = 'Estadual'
    ) as qtd_estadual, (
        SELECT count(e4.co_escola)
        FROM escola e4
        WHERE e4.dependencia_adm = 'Municipal'
    ) as qtd_municipal, (
        SELECT count(e5.co_escola)
        FROM escola e5
        WHERE e5.dependencia_adm = 'Privada'
    ) as qtd_privada
FROM escola e
JOIN distrito d on e.co_distrito = d.co_distrito
JOIN municipio m on d.co_municipio = m.co_municipio
JOIN microrregiao m2 on m.co_microrregiao = m2.co_microrregiao
JOIN mesorregiao m3 on m2.co_mesorregiao = m3.co_mesorregiao
JOIN uf u on m3.co_uf = u.co_uf
JOIN regiao r on u.co_regiao = r.co_regiao
WHERE r.co_regiao = \textcolor{MidnightBlue}{<codigo_regiao>};
        \end{Verbatim}
        
\end{enumerate}


\section{Otimização das Consultas}

Nesta seção iremos analisar o desempenho de cada uma das consultas especificadas na seção 3 (\texttt{Especificação das Consultas}). Utilizando técnicas de indexação e otimização, tentaremos alcançar um aumento significativo no desempenho das consultas.
O sistema de gerenciamento de banco de dados utilizado é o PostgreSQL, versão 9.5. A máquina utilizada para testes conta com um processador i7 e 8GB de memória RAM. As consultas foram executadas cinco vezes.

\subsection{Consulta 1 - Buscar Escolas Disponíveis}

Para fins de teste, utilizaremos os seguintes dados como parâmetros para realização das consultas: 
\begin{itemize}
    \item codigo\_uf = 35 (São Paulo)
    \item codigo\_municipio = 3552205 (Sorocaba)
    \item nome\_escola ILIKE '\%edu\%' (contém no nome o trecho 'edu')
\end{itemize}

\subsubsection{Operador JOIN}

Inicialmente executamos a consulta para obter o tempo de execução.

\vspace{0.5cm}

\begin{flushleft}
\textbf{CONSULTA SQL - JOIN}\\
\end{flushleft}

\begin{Verbatim}[commandchars=\\\{\}]
SELECT  e.co\_escola, e.nome\_escola, e.situacao\_funcionamento,
            e.dependencia\_adm, e.bercario, e.creche, e.pre\_escola,
            e.ens\_fundamental\_anos\_iniciais, e.ens\_fundamental\_anos\_finais,
            e.ens\_medio\_normal, e.ens\_medio\_integrado
FROM escola e
JOIN distrito d on e.co\_distrito = d.co\_distrito
JOIN municipio m on d.co\_municipio = m.co\_municipio
JOIN microrregiao mi on m.co\_microrregiao = mi.co\_microrregiao
JOIN mesorregiao me on mi.co\_mesorregiao = me.co\_mesorregiao
JOIN uf u on me.co\_uf = u.co\_uf
WHERE u.co\_uf = \textcolor{Red}{35}
AND m.co\_municipio = \textcolor{Red}{3552205}
AND e.nome\_escola ILIKE '\textcolor{Red}'
ORDER BY e.qtd\_funcionarios;
\end{Verbatim}

\begin{flushleft}
\textbf{TEMPO DE EXECUÇÃO}\\
\end{flushleft}
30 secs 246 msec.\\
96 rows affected.\\

Conforme o esperado, devido a muitos JOINs, o tempo da consulta foi ruim. Por conseguinte, executamos a mesma consulta com o comando \texttt{EXPLAIN ANALYZE}, obtendo o seguinte plano de execução:


% \vspace{0.5cm}
\begin{flushleft}
\textbf{PLANO DE EXECUÇÃO}\\
\end{flushleft}

\begin{figure}[H]
    \centering
    \includegraphics[width=\textwidth,height=\textwidth,keepaspectratio]{1-join.JPG}
    \caption{Consulta 1 - Original}
    \label{fig:diagrama}
\end{figure}

% EXPLICAÇÃO DO PLANO

Analisando o plano de execução, podemos observar que a \texttt{qtd\_funcionarios} proveniente da tabela \texttt{Escola} foi utilizada para ordenador os resultados obtidos e que o método de ordenação adotado foi o \texttt{quicksort} 

\pagebreak

\subsubsection{Operador IN}

% EXPLICAÇÃO
A consulta inicial utiliza \texttt{JOIN}. Todavia, apesar de obtermos o resultado esperado, o tempo de execução foi consideravelmente alto. Uma medida adotada para obter o mesmo resultado, mas em um tempo de execução menor, foi utilizar \texttt{IN} ao invés de \texttt{JOIN}. 

\vspace{0.5cm}

\begin{flushleft}
\textbf{CONSULTA SQL - IN}\\
\end{flushleft}

\begin{Verbatim}[commandchars=\\\{\}]
SELECT  e.co\_escola, e.nome\_escola, e.situacao\_funcionamento, 
        e.dependencia\_adm, e.bercario, e.creche, e.pre\_escola,
        e.ens\_fundamental\_anos\_iniciais, e.ens\_fundamental\_anos\_finais,
        e.ens\_medio\_normal, e.ens\_medio\_integrado
FROM escola e
WHERE e.co\_distrito IN (
    SELECT d.co\_distrito
    FROM distrito d
    WHERE d.co\_municipio IN (
        SELECT m.co\_municipio
        FROM municipio m
        WHERE m.co\_microrregiao IN (
            SELECT mi.co\_microrregiao
            FROM microrregiao mi
            WHERE mi.co\_mesorregiao IN (
                SELECT me.co\_mesorregiao
                FROM mesorregiao me
                WHERE me.co\_uf = \textcolor{Red}{35}
            )
        )
    )
    AND d.co\_municipio = \textcolor{Red}{3552205}
)
AND e.nome\_escola ILIKE '\textcolor{Red}'
ORDER BY e.qtd\_funcionarios;
\end{Verbatim}

\begin{flushleft}
\textbf{TEMPO DE EXECUÇÃO}\\
\end{flushleft}
19 secs 94 msec.\\
96 rows affected.\\

\pagebreak
\begin{flushleft}
\textbf{PLANO DE EXECUÇÃO}\\
\end{flushleft}

\begin{figure}[H]
    \centering
    \includegraphics[width=\textwidth,height=\textheight,keepaspectratio]{1-in.JPG}
    \caption{Consulta 1 - Operador IN}
    \label{fig:diagrama}
\end{figure}

% EXPLICAÇÃO
É possível notar que ao adotarmos o \texttt{IN} no lugar de \texttt{JOIN} houve uma queda considerável de 11 secs 152 msec, aproximadamente 37\% do tempo inicial. 
\subsubsection{Criação do índice em nome escola}

Uma tentativa para melhorar o desempenho da consulta foi criar um índice para a coluna \texttt{nome\_escola} da tabela \texttt{Escola}:

\begin{verbatim}
    CREATE INDEX nome_escola_index
    ON escola(nome_escola)
    2 secs 936 msec.
\end{verbatim}

Realizamos novamente a mesma consulta com o operador \texttt{IN} e obtemos o seguinte tempo de execução:\\

16 secs 504 msec.

96 rows affected.\\

\pagebreak
\begin{flushleft}
\textbf{PLANO DE EXECUÇÃO}\\
\end{flushleft}

\begin{figure}[H]
    \centering
    \includegraphics[width=\textwidth,height=\textheight,keepaspectratio]{1-index.JPG}
    \caption{Consulta 1 - Criação do Índice \texttt{nome\_escola}}
    \label{fig:diagrama}
\end{figure}

Houve uma queda de aproximadamente 3 segundos em comparação com a consulta anterior, em que não havia o índice. Este valor representa uma diferença de aproximadamente 14\%.

Porém, podemos observar que mesmo com a criação do índice, este não foi utilizado. O otimizador de consultas do PostgreSQL achou melhor não utilizar o índice. Mesmo assim, podemos ver que ele modificou a ordem das operações e conseguiu uma queda de aproximadamente 3 segundos.

\subsubsection{Troca de ILIKE por LIKE}

A última técnica de  otimização utilizada foi trocar o operador \texttt{ILIKE} pelo operador \texttt{LIKE}. O operador \texttt{ILIKE} não considera letras maiúsculas e minúsculas, fato que faz com que haja um aumento no tempo de execução da consulta. 

\vspace{0.5cm}
\begin{flushleft}
\textbf{CONSULTA SQL - LIKE}\\
\end{flushleft}

\begin{Verbatim}[commandchars=\\\{\}]
SELECT  e.co\_escola, e.nome\_escola, e.situacao\_funcionamento, 
        e.dependencia\_adm, e.bercario, e.creche, e.pre\_escola,
        e.ens\_fundamental\_anos\_iniciais, e.ens\_fundamental\_anos\_finais,
        e.ens\_medio\_normal, e.ens\_medio\_integrado
FROM escola e
WHERE e.co\_distrito IN (
    SELECT d.co\_distrito
    FROM distrito d
    WHERE d.co\_municipio IN (
        SELECT m.co\_municipio
        FROM municipio m
        WHERE m.co\_microrregiao IN (
            SELECT mi.co\_microrregiao
            FROM microrregiao mi
            WHERE mi.co\_mesorregiao IN (
                SELECT me.co\_mesorregiao
                FROM mesorregiao me
                WHERE me.co\_uf = \textcolor{Red}{35}
            )
        )
    )
    AND d.co\_municipio = \textcolor{Red}{3552205}
)
AND e.nome\_escola LIKE \textcolor{Red}{'%EDU%'}
ORDER BY e.qtd\_funcionarios;
\end{Verbatim}

\begin{flushleft}
\textbf{TEMPO DE EXECUÇÃO}\\
\end{flushleft}
2 secs 8 msec.\\
96 rows affected.\\




\begin{flushleft}
\textbf{PLANO DE EXECUÇÃO}\\
\end{flushleft}

\begin{figure}[H]
    \centering
    \includegraphics[width=\textwidth,height=\textwidth,keepaspectratio]{1-like.JPG}
    \caption{Consulta 1 - Operador LIKE}
    \label{fig:diagrama}
\end{figure}

Podemos notar que ao realizarmos a mudança dos operadores, houve uma queda aproximadamente 14 segundos em relação ao tempo de execução da consulta anterior. Este tempo representa um diferença de aproximadamente 87\%.   

Apesar do plano de execução dos operadores serem iguais, como o operador \texttt{LIKE} leva em consideração letras minúsculas e maiúsculas, o otimizador conseguiu realizar a consulta muito mais rapidamente.

%-----------------

\subsubsection{Materialized Views}

Outra tentativa de otimização foi realizar a criação de views materializadas, ou seja, que ficam em armazenadas em disco. Deste modo, uma \texttt{MATERIALIZED VIEW} foi criada para cada \texttt{estado} da tabela \texttt{Estado}. 

Para conseguir fazer isto, foi criado um script na linguagem PHP para criar, para cada estado, uma view contendo os códigos dos distritos daquele estado. Foi dado o nome \texttt{distrito<codigo\_estado>}.

Criadas as views, basta realizarmos uma seleção dos códigos dos distritos presentes na view específica do estado desejado, não precisando mais percorrer todos os IN's da consulta anterior.

\vspace{0.5cm}
\begin{flushleft}
\textbf{CONSULTA SQL - COM MATERIALIZED VIEW}\\
\end{flushleft}

\begin{Verbatim}[commandchars=\\\{\}]
SELECT  e.co\_escola, e.nome\_escola, e.situacao\_funcionamento, 
        e.dependencia\_adm, e.bercario, e.creche, e.pre\_escola,
        e.ens\_fundamental\_anos\_iniciais, e.ens\_fundamental\_anos\_finais,
        e.ens\_medio\_normal, e.ens\_medio\_integrado
FROM escola e
WHERE e.co\_distrito in (
    SELECT co\_distrito
    FROM distritos\textcolor{Red}{35} d
    WHERE d.co\_municipio = \textcolor{Red}{3552205}
)
AND e.nome\_escola LIKE \textcolor{Red}{'%EDU%'}
ORDER BY e.qtd\_funcionarios;
\end{Verbatim}

\vspace{0.01cm}
\begin{flushleft}
\textbf{TEMPO DE EXECUÇÃO}\\
\end{flushleft}
1 secs 613 msec.\\

\begin{flushleft}
\textbf{PLANO DE EXECUÇÃO}\\
\end{flushleft}

\begin{figure}[H]
    \centering
    \includegraphics[width=\textwidth,height=\textwidth,keepaspectratio]{1-view.JPG}
    \caption{Consulta 1 - com MATERIALIZED VIEW}
    \label{fig:diagrama}
\end{figure}

Podemos observar que o tempo de execução caiu, porém não significativamente, apenas 395ms. Pelo plano de consulta podemos verificar que o otimizador realizou uma busca sequencial no código do distrito na view de distritos e comparou com o código do distrito na tabela escola. Por fim, realizou uma ordenação com quicksort pela quantidade de funcionários.


\subsubsection{Criação do Índice em Escola(Código Distrito)}

A última otimização realizada foi a criação de um índice na coluna \texttt{co\_distrito} da tabela \texttt{Escola}. A SQL utilizada para a criação do índice e o tempo de execução podem ser observados a seguir:

\begin{verbatim}
CREATE INDEX e_co_distrito_index 
ON escola(co_distrito)
3 secs 607 msec.
\end{verbatim}

Realizando novamente a mesma consulta do item anterior, observamos os seguintes resultados:

\vspace{0.01cm}
\begin{flushleft}
\textbf{TEMPO DE EXECUÇÃO}\\
\end{flushleft}
482 msec.\\

\begin{flushleft}
\textbf{PLANO DE EXECUÇÃO}\\
\end{flushleft}

\begin{figure}[H]
    \centering
    \includegraphics[width=\textwidth,height=\textwidth,keepaspectratio]{1-index-co-distrito.JPG}
    \caption{Consulta 1 - com Índice no Código do Distrito}
    \label{fig:diagrama}
\end{figure}

Como podemos observar, o tempo de execução caiu para apenas 482 ms! Uma diminuição de 70\% do valor anterior, que ainda não continha o índice. Fica clara esta diferença ao visualizar o plano de execução, uma vez que o otimizador utilizou o novo índice criado.


\subsubsection{Conclusão - Consulta 1}
\vspace{0.5cm}
Na Tabela 1 podemos observar as diferenças, em porcentagem, dos tempos de execução de cada um dos testes realizados para a Consulta 1, e observar que conseguimos uma melhora final de 98,4\% em relação ao tempo inicial:

\begin{table}[htbp]
  \centering
  \caption{Comparação Consulta 1}
    \resizebox{\textwidth}{!}{\begin{tabular}{|l|c|c|c|}
    \hline
    \toprule
    \textbf{Testes} & \textbf{Tempo (ms)} & \textbf{Diferença com Anterior (\%)} & \multicolumn{1}{l|}{\textbf{Diferença com Original (\%)}} \\
    \hline
    \midrule
    Consulta Inicial (JOIN) & 30246 & -     & - \\
    \midrule
    IN    & 19094 & 36,87 & 36,87 \\
    \midrule
    ÍNDICE & 16504 & 13,56 & 45,43 \\
    \midrule
    LIKE  & 2008  & 87,83 & 93,36 \\
    \midrule
    MATERIALIZED VIEW & 1613  & 19,67 & 94,66 \\
    \midrule
    ÍNDICE (co\_distrito) & 482   & 70,11 & 98,4 \\
    \bottomrule
    \hline
    \end{tabular}}%
  \label{tab:compconsulta1}%
\end{table}%


\subsection{Consulta 2 - Quantidade de Escolas de cada Dependência Administrativa}

A segunda consulta também foi inicialmente realizada com o operador \texttt{JOIN}. Nesta consulta foram utilizadas subconsultas correlatas na cláusula do \texttt{SELECT} para que fosse possível recuperar as quantidades de cada dependência de uma região em uma única consulta. Para fins de teste, utilizaremos o seguinte dado como parâmetro para realização das consultas: 
\begin{itemize}
    \item codigo\_regiao = 3 (Sudeste)
\end{itemize}

\subsubsection{Com JOIN}

\vspace{0.5cm}
\begin{flushleft}
\textbf{CONSULTA SQL - JOIN}\\
\end{flushleft}

\begin{Verbatim}[commandchars=\\\{\}]
SELECT count(e.co_escola) as qtd_escolas, (
        SELECT count(e2.co_escola)
        FROM escola e2
        WHERE e2.dependencia_adm = \textcolor{Red}{'Federal'}
    ) as qtd_federal, (
        SELECT count(e3.co_escola)
        FROM escola e3
        WHERE e3.dependencia_adm = \textcolor{Red}{'Estadual'}
    ) as qtd_estadual, (
        SELECT count(e4.co_escola)
        FROM escola e4
        WHERE e4.dependencia_adm = \textcolor{Red}{'Municipal'}
    ) as qtd_municipal, (
        SELECT count(e5.co_escola)
        FROM escola e5
        WHERE e5.dependencia_adm = \textcolor{Red}{'Privada'}
    ) as qtd_privada
FROM escola e
JOIN distrito d on e.co_distrito = d.co_distrito
JOIN municipio m on d.co_municipio = m.co_municipio
JOIN microrregiao m2 on m.co_microrregiao = m2.co_microrregiao
JOIN mesorregiao m3 on m2.co_mesorregiao = m3.co_mesorregiao
JOIN uf u on m3.co_uf = u.co_uf
JOIN regiao r on u.co_regiao = r.co_regiao
WHERE r.co_regiao = \textcolor{Red}{3};
\end{Verbatim}

\begin{flushleft}
\textbf{TEMPO DE EXECUÇÃO}\\
\end{flushleft}
2 secs 522 msec.\\
1 rows affected.\\

\begin{flushleft}
\textbf{PLANO DE EXECUÇÃO}\\
\end{flushleft}

\begin{figure}[H]
    \centering
    \includegraphics[width=\textwidth,height=\textwidth,keepaspectratio]{2-join.JPG}
    \caption{Consulta 2 - Original}
    \label{fig:diagrama}
\end{figure}

Podemos observar que o plano de execução foi realizado baseado em agregação, ou seja, foi realizada uma agregação do retorno de cada subconsulta correlata com a junção entre todas as tabelas para recuperar os dados da região especificada. O retorno de cada subconsulta correlata foi armazenado em uma variável, como se cada uma destas tivesse seu próprio plano de execução (InitPlan 1, 2, 3 e 4).

Apesar de todas as junções e agregações, o tempo de execução foi relativamente rápido, apenas 2 segundos e 522 milissegundos.

\subsubsection{Com IN}

Com intuíto de diminuir o tempo de execução modificamos a consulta para a utilizarmos o operador \texttt{IN} ao invés do operador \texttt{JOIN}, embora este já fosse bem pequeno.

\vspace{0.5cm}

\begin{flushleft}
\textbf{CONSULTA SQL - IN}\\
\end{flushleft}

\begin{Verbatim}[commandchars=\\\{\}]
SELECT count(e.co_escola) as qtd_escolas, (
        SELECT count(e2.co_escola)
        FROM escola e2
        WHERE e2.dependencia_adm = \textcolor{Red}{'Federal'}
    ) as qtd_federal, (
        SELECT count(e3.co_escola)
        FROM escola e3
        WHERE e3.dependencia_adm = \textcolor{Red}{'Estadual'}
    ) as qtd_estadual, (
        SELECT count(e4.co_escola)
        FROM escola e4
        WHERE e4.dependencia_adm = \textcolor{Red}{'Municipal'}
    ) as qtd_municipal, (
        SELECT count(e5.co_escola)
        FROM escola e5
        WHERE e5.dependencia_adm = \textcolor{Red}{'Privada'}
    ) as qtd_privada
FROM escola e
WHERE e.co_distrito IN (
    SELECT d.co_distrito
    FROM distrito d
    WHERE d.co_municipio IN (
        SELECT m.co_municipio
        FROM municipio m
        WHERE m.co_microrregiao IN (
            SELECT mi.co_microrregiao
            FROM microrregiao mi
            WHERE mi.co_mesorregiao IN (
                SELECT me.co_mesorregiao
                FROM mesorregiao me
                WHERE me.co_uf IN (
                    SELECT u.co_uf
                    FROM uf u
                    WHERE u.co_regiao = \textcolor{Red}{3}
                )
            )
        )
    )
);
\end{Verbatim}

\begin{flushleft}
\textbf{TEMPO DE EXECUÇÃO}\\
\end{flushleft}
1 secs 679 msec.\\
1 rows affected.\\

\begin{flushleft}
\textbf{PLANO DE EXECUÇÃO}\\
\end{flushleft}

\begin{figure}[H]
    \centering
    \includegraphics[width=\textwidth,height=\textwidth,keepaspectratio]{2-in.JPG}
    \caption{Consulta 2 - Operador IN}
    \label{fig:diagrama}
\end{figure}

Podemos observar que o tempo de execução sofreu uma redução de aproximadamente 1 segundo. Queda significava se observarmos que este valor representa aproximadamente 33\% do original.

\pagebreak
Na Tabela 2 podemos observar as comparações entre os tempos de execução de cada um dos testes realizados para a consulta 2.

\begin{table}[htbp]
  \centering
  \caption{Comparação Consulta 2}
    \resizebox{\textwidth}{!}{\begin{tabular}{|l|c|c|c|}
    \hline
    \toprule
    \textbf{Testes} & \textbf{Tempo (ms)} & \textbf{Diferença com Anterior (\%)} & \multicolumn{1}{l|}{\textbf{Diferença com Original (\%)}} \\
    \hline
    \midrule
    Consulta Inicial (JOIN) & 2522  & -     & - \\
    \midrule
    IN    & 1679  & 33,55 & 33,55 \\
    \bottomrule
    \hline
    \end{tabular}}%
  \label{tab:compconsulta2}%
\end{table}%



\end{document}